\documentclass{article}
\usepackage{amsthm}
\usepackage{amsmath}
\usepackage{amssymb}
\usepackage{multicol}
\usepackage{graphicx}
\usepackage{epstopdf}

\theoremstyle{definition}
\newtheorem{mydef}{Definition}
\newtheorem{myex}{Example}
\newtheorem{mythe}{Theorem}
\newtheorem{mylemma}{Lemma}
\newtheorem{mycor}{Corollary}
\newtheorem*{myab}{Abstract}
\begin{document}

\title{Determining Equivalence Classes}
\maketitle

\section{Basics}

To convert from $\mathcal{G}$ to $\mathcal{G}^u$, we use the following:
\\
1. $V_i \rightarrow V_j$ in $\mathcal{G}^u$ iff $\exists$ a directed sequence from $V_i$ to $V_j$ of length $u$ in $\mathcal{G}$
\\
2. $V_i \leftrightarrow V_j$ in $\mathcal{G}^u$ iff $\exists$ a pair of directed sequences:
\\
$ \pi_1$:$ V_c \rightarrow ... \rightarrow V_i$
\\
$ \pi_2$:$ V_c \rightarrow ... \rightarrow V_j$
\\
where length($\pi_1$)=length($\pi_2$) $\leq u$ and $V_c$ is some vertex in $\mathcal{G}$ 
\\
\\
The pair of directed sequences in step $2$ forms a \textit{fork}. Also $\mathcal{G}= \mathcal{G}^1$. No bidirected edges are allowed in $\mathcal{G}^1$. Also note that if $u \rightarrow v$ and $v \rightarrow u$, this is not the same as $u \leftrightarrow v$. Sergey considers these separate. The former is two single directed edges whereas the latter is one bidirected edge. It can be the case that there exists a bidirected edge between $u$ and $v$: $u \leftrightarrow v$ and only a single directed edge from $u$ to $v$: $u \rightarrow v$.
\\
\\
Converting from $\mathcal{G}^1$ to $\mathcal{G}^u$ is called the \textit{forward problem}. Given a $\mathcal{H}$, finding matching $\mathcal{G}$'s and $u$ satisfying $\mathcal{H} = \mathcal{G}^u$ is called the \textit{backward problem}. 
\\
\\
Although we can extend $u$ indefinitely for $\mathcal{G}$, there exists a $u_{MAX}$ such that for any $u \geq u_{MAX}$, $\mathcal{G}^{u}$ is a graph in $\{\mathcal{G}^2,\mathcal{G}^3,...,\mathcal{G}^{u_{MAX}}\}$
\\
\\
The equivalence class of a graph $\mathcal{H}$ (symbolized $[]_{\mathcal{H}}$) consists of all ground truths $\mathcal{G}$ such that $\mathcal{G}^u = \mathcal{H}$ where $u$ is ANY undersampling rate.

\newpage

\section{Question}

Given an $\mathcal{H}$ and a $\mathcal{G'}$ that belongs in the equivalence class of $\mathcal{H}$, is it possible to determine ALL elements of this equivlance class efficiently?





	


%\begin{figure}[!h]
%\includegraphics[scale=.28]{name_of_pic}
%\end{figure}


\end{document}
%$\mathcal{G}^u$